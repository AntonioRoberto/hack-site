% Appendix Template

\chapter{Juízes Online (Online Judges)} % Main appendix title

\label{AppendixB} % Change X to a consecutive letter; for referencing this appendix elsewhere, use \ref{AppendixX}

Online Judges são plataformas online que contam com um banco de dados com diversos tipos de problemas de competições de programação, e com um sistema de correção online.

Para afirmar que sua solução está correta, basta enviar o código fonte da sua solução (em geral escrito em C++ ou JAVA) para uma dessas plataformas.

Alguns desses Online Judges são citados em seguida.

\section{UVa} 
	
Criado em 1995 pelo matemático Miguel Ángel Revilla, é atualmente um dos Online Judges mais famoso entre os participantes da ACM-ICPC.

É hospedado pela \href{http://www.uva.es/export/sites/uva/}{Universidade de Valhadolide} e conta com mais de 100000 usuários registrados.

Site: \href{https://uva.onlinejudge.org/}{https://uva.onlinejudge.org/}

\section{Topcoder} 

Empresa que administra competições de programação nas linguagens Java, C++ e C$\#$.

É responsável também por aplicar competições de design e desenvolvimento de software.

Site: \href{https://www.topcoder.com/}{https://www.topcoder.com/}

\section{Codeforces}

Site Russo dedicado competições de programação. 

Em 2013, Codeforces superou Topcoder com relação ao número de usuários ativos, apesar de ter sido criado quase 10 anos depois.

O estilo de problemas que esse site aplica é similar aos problemas encontrados na ACM-ICPC.

Site: \href{http://codeforces.com/}{http://codeforces.com/}

\section{CodeChef}

Iniciativa educacional sem fins lucrativos lançada em 2009 pela \href{http://www.directi.com/}{Direct}.

É uma plataforma de progamação competitiva que suporta mais de 35 linguagens de programação.

Site: \href{https://www.codechef.com/}{https://www.codechef.com/}

