% Chapter 2

\chapter{Congruência} % Main chapter title

\label{Chapter2} % Change X to a consecutive number; for referencing this chapter elsewhere, use \ref{ChapterX}

%----------------------------------------------------------------------------------------
%	SECTION
%----------------------------------------------------------------------------------------

\section{Congruência}



%----------------------------------------------------------------------------------------
%	SECTION
%----------------------------------------------------------------------------------------

\section{Congruência Linear}



%----------------------------------------------------------------------------------------
%	SECTION
%----------------------------------------------------------------------------------------

\section{Teorema de Fermat, Euler e Wilson}



%----------------------------------------------------------------------------------------
%	SECTION
%----------------------------------------------------------------------------------------

\section{Teorema do Resto Chinês}

\begin{theorem}[Teorema do Resto Chinês]
Tome o sistema de congruências lineares:\\

$a_1x \equiv c_1 (mod$ $m_1)$\\
$a_2x \equiv c_2 (mod$ $m_2)$\\
$a_3x \equiv c_3 (mod$ $m_3)$\\
$...$\\
$a_nx \equiv c_n (mod$ $m_n)$\\

Em que $c_i \in \mathbb{Z}$, $MDC(a_i,m_i) = 1$, e $MDC(m_i, m_j) = 1$ para $i \neq j$
Nessas condições o sistema acima tem solução única módulo $M$, em que $M = m_1m_2m_3...m_n$.
\end{theorem}
\textbf{Demonstração:}
Deixaremos a demostração a cargo do leitor.

%----------------------------------------------------------------------------------------
%	SECTION
%----------------------------------------------------------------------------------------

\section{Problemas Propostos}



%----------------------------------------------------------------------------------------
\subsection{UVA-10090}
\href{https://uva.onlinejudge.org/index.php?option=onlinejudge&page=show_problem&problem=1031}{10090 - Marbles}\\


\textbf{Resumo:}
É dado um número inteiro $n$ ($0 < n \leq 10^8$). O problema consite em verificar se $n$ pode, ou não pode, ser escrito como a soma de dois números primos.
E em caso afirmativo encontrar o valor desses dois primos.
\\

\textbf{Solução:}
\\

\textbf{Pseudocódigo:}
\begin{algorithm}
\caption{Marbles}
\begin{algorithmic}[1]
\Procedure{FindTwoPrimesSum (n)}{}

\EndProcedure
\end{algorithmic}
\end{algorithm}


\textbf{Análise:}


