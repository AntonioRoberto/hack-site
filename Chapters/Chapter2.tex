% Chapter 2

\chapter{Aritmética Modular} % Main chapter title

\label{Chapter2} % Change X to a consecutive number; for referencing this chapter elsewhere, use \ref{ChapterX}

%----------------------------------------------------------------------------------------
%	SECTION
%----------------------------------------------------------------------------------------

\section{Congruência}

\begin{definition}
Para $a$ e $b$ inteiros, dizemos que $a$ é congruente à $b$ módulo $m$ ($a \equiv b (mod$ $m), m > 0$) 
se $a$ e $b$ produzem o mesmo resto na divisão por $m$ (ie, $m|(a-b)$).
Caso contrário ($m\nmid (a-b)$), dizemos que $a$ não é congruente à $b$ módulo $m$ ($a \not\equiv b (mod $ $m)$).
\\
\end{definition}


\begin{definition}
Dizemos que o conjunto de inteiros $S = \{s_1, s_2, ..., s_k\}$ é um sistema completo de resíduos modulo $n$ se:
%\begin{enumerate}
$\forall a \in \mathbb{Z}, \exists! s_i \in S \mid a \equiv s_i (mod$ $n)$
%\item$s_i \not\equiv s_j (mod $ $n)$ para $i \neq j$
%\\
%\end{enumerate}
\end{definition}


\begin{proposition}\label{corolario_implicacao_mdc}
Dados inteiros $a$, $b$, $c$, $d$ com $MDC(c,d)=1$, temos que: 

$ac \equiv bc (mod$ $d) \Rightarrow a \equiv b (mod$ $d)$.
\end{proposition}
\textbf{Demonstração:}

$ac \equiv bc (mod$ $d) \Rightarrow d|(ac-bc) \Rightarrow d|c(a-b) \Rightarrow d|(a-b)$, já que $MDC(c,d)=1$

$\Rightarrow a \equiv b (mod$ $d)$. $\square$
\\


\begin{proposition}\label{residuo_completo_1n}
O conjunto $R = \{0, 1, 2, 3,...,n-1\}$, é um sistema completo de resíduos módulo $n$.
\end{proposition}
\textbf{Demonstração:}
Pelo \autoref{algoritmo_divisao} sabemos que para qualquer inteiro $a$, existe $q, r$ tal que, $a = qn + r, 0 \leq r < n$. Assim, $a\equiv r(mod$ $n)$, com $r \in R$. $\square$
\\

\begin{theorem}
Se o conjunto $S = \{s_0, s_1, s_2, ..., s_{k-1}\}$ é um sistema completo de resíduos módulo $n$, então $k=n$.
\end{theorem}
\textbf{Demonstração:}
Tome o conjunto $R = \{0, 1, 2, 3,...,n-1\}$. Pela \textbf{Proposição} \autoref{residuo_completo_1n} sabemos que $R$ é um sistema completo de resíduos módulo $n$.

Podemos concluir então, que cada elemento $s_i$ de $S$ é congruente a exatamente um dos elementos $r_i$ em $R$, o que nos garante $|S| \leq |R|$, ou simplesmente $k\leq n$. 
Por outro lado, o conjunto $S$ é por definição um sistema completo de resíduos módulo $n$, e desse modo cada elemento $r_i$ de $R$ é congruente a exatamente um dos elementos $s_i$ em $S$, o que nos garante $|R| \leq |S|$.
Disso temos que, $|R| = |S|$, ou melhor, $k=n$. $\square$
\\



%----------------------------------------------------------------------------------------
%	SECTION
%----------------------------------------------------------------------------------------

\section{Congruência Linear}

\begin{definition}
Congruências da forma $ax \equiv b (mod$ $m)$, onde $a$, $b$ e $m$ são inteiros e $x$ é uma incógnita, são chamadas de \textit{Congruências Lineares}.
\end{definition}



\begin{proposition}\label{congruencia_linear_ida}
$ax \equiv b (mod$ $m)$ tem solução $\Rightarrow$ $MDC(a,m)|b$
\end{proposition}
\textbf{Demonstração:}
Suponha que $\exists x\in\mathbb{Z}$, tal que $ax\equiv b(mod$ $m)$, assim temos:

$ax\equiv b(mod$ $m) \Rightarrow m|(b-ax) \Rightarrow \exists! r\in\mathbb{Z}$ tal que $(b-ax) = mr$

$\Rightarrow b = mr + ax \Rightarrow MDC(a,m)|b$, pois $MDC(a,m)$ divide tanto $a$ como $m$. $\square$



\begin{proposition}\label{congruencia_linear_volta}
$ax \equiv b (mod$ $m)$ tem solução $\Leftarrow$ $MDC(a,m)|b$
\end{proposition}
\textbf{Demonstração:}

$MDC(a,m)|b \Rightarrow (ax+my)|b$ ($\triangleright$ \autoref{teorema_bezout})

$\Rightarrow \exists!r\in\mathbb{Z}$ tal que, $b=(ax + my)r \Rightarrow b = (xr)a + (yr)m$

$\Rightarrow b \equiv xra + yrm$ $(mod$ $m) \Rightarrow b \equiv xra$ $(mod$ $m)$

E assim, a \textbf{Congruência Linear} $az\equiv b (mod$ $m)$, tem solução $z=xr$. $\square$

\begin{corollary}\label{teorema_congruencia_linear_ida_volta}
$ax \equiv b (mod$ $m)$ tem solução $\Leftrightarrow$ $MDC(a,m)|b$
\end{corollary}
\textbf{Demonstração:}
Segue trivialmente das \textbf{Proposições} \autoref{congruencia_linear_ida} e \autoref{congruencia_linear_volta}.




\begin{theorem}\label{sistemo_completp_residuo}
O conjunto $S = \{s_0, s_1, s_2,..., s_{n-1}\}$ com $s_i = im + p$, $m, n, p \in \mathbb{Z}$, e $MDC(m,n) = 1$ é um sistema completo de resíduos módulo $n$.
\end{theorem}
\textbf{Demonstração:}
Primeiro provaremos que $\forall a \in \mathbb{Z}, \exists! s_i \in S \mid a \equiv s_i (mod$ $n)$.

Tome um inteiro $a$ qualquer e a Congruência Linear: $(a-p)\equiv xm(mod$ $n)$. Pelo \textbf{Corolário} \autoref{teorema_congruencia_linear_ida_volta} sabemos que essa Congruência Linear tem solução, já que $MDC(m,n)=1$. Assim:

$(a-p)\equiv xm(mod$ $n) \Rightarrow a\equiv xm+p(mod$ $n) \Rightarrow a\equiv im + p (mod$ $n), i=x\bmod n$

$\Rightarrow a\equiv s_i(mod$ $n)$

Agora provaremos que $s_i \not\equiv s_j (mod $ $n)$ para $i \neq j$.

Tome $s_i$ e $s_j$ em $S$ com $i\neq j$, $0 < |i-j| < n$. Claramente $(i-j)\not\equiv 0(mod$ $n)$, já que que $i$ e $j$ são distintos e $0 \leq i,j < n$.
Portanto $(i-j)m\not\equiv 0(mod$ $n)$, já que $MDC(m,n)=1$, e assim:

$(i-j)m\not\equiv 0(mod$ $n) \Rightarrow im \not\equiv jm(mod$ $n) \Rightarrow im+p \not\equiv jm+p(mod$ $n)$

$\Rightarrow s_i \not\equiv s_j(mod$ $n)$.

Disso segue que $S$ é um sistema completo de resíduos módulo $n$. $\square$


%----------------------------------------------------------------------------------------
%	SECTION
%----------------------------------------------------------------------------------------

\section{Teoremas de Fermat e do Resto Chinês}

\subsection{Teorema de Fermat}

\begin{theorem}[Pequeno Teorema de Fermat]\label{teorema_fermat}
Dado um número primo qualquer $p$, temos que: 
$a^{p-1} \equiv 1 (\bmod$ $p), \forall a \in \mathbb{Z} \mid MDC(a, p) = 1$
\end{theorem}
\textbf{Demonstração:}
Tome os conjuntos $S = \{s_0, s_1, s_2, ..., s_{p-1}\}$, com $s_i = ai$, e $T = \{0, 1, 2,..., p-1\}$. 
Claramente o conjunto $T$ é um \textit{Sistema completo de resíduos módulo $p$}.
Pelo \autoref{sistemo_completp_residuo} sabemos que $S$ também é um \textit{Sistema completo de resíduos módulo $p$},
e assim, $\forall s_i\in S$, $\exists! t_j\in T, 0\leq t_j\leq(p-1)$, tal que $s_i\equiv t_j(mod$ $p)$. Dessa informação podemos derivar a seguinte congruência modular:

$s_1.s_2.s_3...s_{p-1} \equiv t_1.t_2.t_3...t_{p-1} (mod$ $p)$ ($\triangleright$ Observe que o correspondente a $s_0$ é $t_0$)

$\Rightarrow a.2a.3a....(p-1)a \equiv 1.2.3....(p-1) (mod$ $p) \Rightarrow a^{p-1}(p-1)! \equiv (p-1)! (mod$ $p)$

E aplicando os \textbf{Corolários} \autoref{mdc_primo_fatorial} e \textbf{Proposição} \autoref{corolario_implicacao_mdc}, temos:

$a^{p-1}\equiv 1(mod$ $p)$. $\square$
\\


\begin{theorem}\label{teorema_fermat_expansao}
Dados os inteiros $a$ e $b$ quaisquer e um número primo $p$, com $MDC(a, p) = 1$, temos que:

$a^{b} \equiv a^{b \bmod (p-1)} (\bmod$ $p)$

\end{theorem}
\textbf{Demonstração:}
Pelo \autoref{algoritmo_divisao} podemos escrever $b=q(p-1)+r$, onde $r=b\bmod(p-1)$. Assim temos:

$a^b = a^{q(p-1)+r} = a^{q(p-1)}a^r = (a^{p-1})^qa^r \Rightarrow a^b \equiv (a^{p-1})^qa^r (mod$ $p)$ 

Pelo \autoref{teorema_fermat} temos $a^{p-1} \equiv 1 (\bmod$ $p)$. Logo:

$a^b \equiv (1)^qa^r \equiv a^r \equiv a^{b \bmod (p-1)} (mod$ $p)$. $\square$
\\


\subsection{Teorema do Resto Chinês}

\begin{theorem}[Teorema do Resto Chinês]
Tome o sistema de congruências lineares:

$a_1x \equiv c_1 (mod$ $m_1)$

$a_2x \equiv c_2 (mod$ $m_2)$

$a_3x \equiv c_3 (mod$ $m_3)$

$...$

$a_nx \equiv c_n (mod$ $m_n)$\\

Em que $c_i \in \mathbb{Z}$, $MDC(a_i,m_i) = 1$, e $MDC(m_i, m_j) = 1$ para $i \neq j$
Nessas condições o sistema acima tem solução única módulo $M$, em que $M = m_1m_2m_3...m_n$.
\end{theorem}
\textbf{Demonstração:}
TODO


%----------------------------------------------------------------------------------------
%	SECTION
%----------------------------------------------------------------------------------------

\section{Exponenciação Modular}

\textit{Exponenciação Modular} é um algoritmo muito usado em \textit{Ciência da Computação} principalmente no campo da \textit{Criptografia}.

O algoritmo recebe inteiros $a$, $b$, e $m$ e calcula $a^b\bmod m$, usando divisão e conquista sobre a seguinte equação:
\[
 a^b\bmod m = 
  \begin{cases} 
   (a^{\lfloor \frac{b}{2} \rfloor})^2 \mod m& \text{se } b \text{ for par} \\
   a(a^{\lfloor \frac{b}{2} \rfloor})^2 \mod m& \text{se } b \text{ for impar} 
  \end{cases}
\]
\clearpage

\textbf{Pseudocódigo:}
\begin{algorithm}
\caption{Exponenciação Modular}
\begin{algorithmic}[1]
\Procedure{$EXPMOD (a, b, m)$}{}
\If {$b = 0$}
\State \Return $1$
\EndIf 
\\
\State $pot \gets EXPMOD(a, \lfloor \frac{b}{2} \rfloor, m)$
\State $pot \gets pot^2 \mod m$
\\
\If {$b \equiv 0 (mod$ $2)$}
\State \Return $pot$
\Else
\State \Return $a(pot) \mod m$
\EndIf

\EndProcedure
\end{algorithmic}
\end{algorithm}


\textbf{Análise:}
O tempo $T(a,b,m)$ que o algoritmo consome é dado por: $T(a,b,m) = T(a,b/2,m)+O(1)$.

Assim temos que a complexidade do algoritmo é $O(\log b)$.


%----------------------------------------------------------------------------------------
%	SECTION
%----------------------------------------------------------------------------------------

\section{Problemas Propostos}



%----------------------------------------------------------------------------------------
\subsection{UVA-10090}
\href{https://uva.onlinejudge.org/index.php?option=onlinejudge&page=show_problem&problem=1031}{10090 - Marbles}\\


\textbf{Resumo:}
É dado um número inteiro $n$ ($0 < n \leq 10^8$). O problema consite em verificar se $n$ pode, ou não pode, ser escrito como a soma de dois números primos.
E em caso afirmativo encontrar o valor desses dois primos.
\\

\textbf{Solução:}
\\

\textbf{Pseudocódigo:}
\begin{algorithm}
\caption{Marbles}
\begin{algorithmic}[1]
\Procedure{FindTwoPrimesSum (n)}{}

\EndProcedure
\end{algorithmic}
\end{algorithm}


\textbf{Análise:}



%----------------------------------------------------------------------------------------
\subsection{CodeChef-IITK2P10}
\href{https://www.codechef.com/problems/IITK2P10}{IITK2P10 - Chef and Pattern}\\


\textbf{Resumo:}
Tome a seguinte função $f_K:\mathbb{N}^* \longmapsto \mathbb{N}$:

\[
 f_K(x) = 
  \begin{cases} 
   1 & \text{se } x = 1 \\
   K & \text{se } x = 2 \\
   \prod_{i=1}^{x-1}f_K(i) & \text{se } x \geq 3
  \end{cases}
\]

São dados dois números inteiros $N$, $K$ ($1 \leq N \leq 10^9$, $1 \leq K \leq 10^5$). O problema consiste em calcular a expressão: $f_K(N) \bmod p$, em que $p = (10^9+7)$.
\\

\textbf{Solução:}
Escrevendo os valores dos primeiros termos que a função assume, temos: $f_K(1)=1, f_K(2)=K, f_K(3)=K, f_K(4)=K^2, f_K(5)=K^4, f_K(6)=K^8, f_K(7)=K^{16}$.

Provaremos, por indução, que $f_K(N) = K^{2^{N-3}}, N \geq 3$.

Para os primeiros termos essa expressão é trivialmente verificada.

Assuma que a expressão funciona para algum número natural qualquer $(R-1) \geq 3$ ($f_K(R-1) = K^{2^{R-4}}$).

Nessas condições temos que: 

$f_K(R) = \prod_{i=1}^{R-1}f_K(i) = 1.K.\prod_{i=3}^{R-1}f_K(i) = K\prod_{i=3}^{R-1}K^{2^{i-3}} = K\prod_{j=0}^{R-4}K^{2^j}$

$f_K(R) = KK^{\sum_{j=0}^{R-4}2^j} = KK^{2^{R-3}-1} = K^{2^{R-3}}$ $\square$

Para calcular o valor de $f_K(N) \bmod p$, podemos aplicar o \autoref{teorema_fermat_expansao}, já que $p$ é um número primo e $MDC(p, K) = 1$:

$f_K(N) \bmod p = K^{2^{N-3}} \bmod p = K^{2^{N-3} \bmod (p-1)} \bmod p$ 

Reduzindo o problema, dessa maneira, em calcular: $K^{2^{N-3}} \bmod (10^9+7)$.
\\

\textbf{Pseudocódigo:}
\begin{algorithm}
\caption{Chef and Pattern}
\begin{algorithmic}[1]
\Procedure{f (N, K)}{}
\State $p \gets (10^9+7)$
\State $exp \gets EXPMOD(2, N-3, p-1)$ \Comment{$exp = 2^{N-3} \bmod (p-1)$}
\State $solution \gets EXPMOD(K, exp, p)$ \Comment{$solution = K^{2^{N-3} \bmod (p-1)} \bmod p$}
\State \Return{$solution$}
\EndProcedure
\end{algorithmic}
\end{algorithm}

\textbf{Análise:}
Como vimos anteriormente, as linhas $3$ e $4$ do algoritmo consomem tempo proporcional à $O(n \log n)$, e assim a complexidade total é $O(n \log n)$.

