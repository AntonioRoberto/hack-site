% Chapter 3

\chapter{Funções Aritméticas} % Main chapter title

\label{Chapter3} % Change X to a consecutive number; for referencing this chapter elsewhere, use \ref{ChapterX}

%----------------------------------------------------------------------------------------
%	SECTION
%----------------------------------------------------------------------------------------

\section{$\Phi$ de Euler}

\begin{definition}
A Função Totiente de Euler, denotada por $\Phi(n)$, é a função aritmética que conta o número 
de inteiros positivos menores ou iguais a $n$ que são primos entre si com $n$.

$\Phi(n) := |\{ x \in \mathbb{N}^{*} \mid MDC(x,n) = 1 \}|$
\end{definition}



\begin{theorem}\label{phi_multiplicativa}
$\Phi(n)$ é função multiplicativa, ie, $\Phi(mn) = \Phi(m)\Phi(n)$ para $MDC(m,n) = 1$.
\end{theorem}
\textbf{Demonstração:}
A demonstração asseguir foi retirada livro: \textit{OLIVEIRA SANTOS, José Plínio de. Introdução à Teoria dos Números. IMPA, 1998. 72 p.}

Vamos dispor os números de 1 até $mn$ da seguinte forma:\\

$1\quad(m+1)\quad(2m+1)\quad...\quad(n-1)m+1$

$2\quad(m+2)\quad(2m+2)\quad...\quad(n-1)m+2$

$3\quad(m+3)\quad(2m+3)\quad...\quad(n-1)m+3$

$\vdots$

$m\quad\quad2m\quad\quad\quad3m\quad\quad...\quad\quad\quad\quad{nm}$\\

Se na linha $r$, onde estão os termos $r,m+r,2m+r,...,(n-1)m+r$, tivermos $MDC(m,r) = d > 1$, então nenhum termo nesta linha será primo com $mn$, uma vez que estes termos, sendo da forma $km+r, 0 \leq k \leq n-1$, são todos divisíveis por $d$ que é o 
\textbf{Máximo Divisor Comum} de $m$ e $r$. Logo, para encontrarmos os inteiros desta tabela que são primos com $mn$, devemos olhar na linha $r$ somente se $MDC(m,r)=1$.
Portato temos $\Phi(m)$ linhas onde todos os elementos são primos com $m$.

Devemos, pois, procurar em cada uma dessas $\Phi(m)$ linhas, quantos elementos são primos com $n$, uma vez que todos são primos com $m$. Como $MDC(m,n)=1$ os elementos $r,m+r,2m+r,...,(n-1)m+r$ formam um sistema completo de resíduos módulo $n$ (\autoref{sistemo_completp_residuo}).
Logo, cada uma destas linhas possui $\Phi(n)$ elementos primos com $n$ e, portanto, como eles são primos com $m$, eles são primos com $mn$. Isto nos garante que $\Phi(m) = \Phi(m)\Phi(n)$. $\square$


\begin{theorem}\label{phi_potencia}
$\Phi(p^k) = (p^k - p^{k-1})$, para $p$ primo e $k$ inteiro positivo.
\end{theorem}
\textbf{Demonstração:}
Como $p$ é um número primo, para qualquer inteiro $n$, os únicos valores possíveis para $MDC(p^k,n)$ são: $1, p, p^2,...,p^k$, 
e desse modo, se $MDC(p^k,n)\neq1$ temos que $p|n$ ($n$ é múltiplo de $p$). Assim, o quantidade de números não-primos e menores do que $p^k$ é $p^{k-1}$.

Logo, temos que: $\Phi(p^k) = p^k - p^{k-1}$ $\square$




\begin{theorem}[Fórmula Produto de Euler]
$\Phi(n) = n \prod_{p|n}(1 - \frac{1}{p}) = n \prod_{p|n}(\frac{p-1}{p})$
\end{theorem}
\textbf{Demonstração:}

$\Phi(n) = \Phi(p_1^{a_1}p_2^{a_2}...p_k^{a_k})$ ($\triangleright$ \autoref{fatoracao_unica})

$\Phi(n) = \Phi(p_1^{a_1})\Phi(p_2^{a_2})...\Phi(p_k^{a_k})$ ($\triangleright$ \autoref{phi_multiplicativa})

$\Phi(n) = (p_1^{a_1} - p_1^{a_1-1})(p_2^{a_2} - p_2^{a_2-1})...(p_k^{a_k} - p_k^{a_k-1})$ ($\triangleright$ \autoref{phi_potencia})

$\Phi(n) = p_1^{a_1}p_2^{a_2}...p_k^{a_k}(1 - 1/p_1)(1 - 1/p_2)...(1 - 1/p_k)$

$\Phi(n) = n \prod_{p|n}(1 - \frac{1}{p})$ $\square$\\

\textbf{Pseudocódigo:}
\begin{algorithm}
\caption{Calcula os primeiros N termos da função $\Phi$}
\begin{algorithmic}[1]
\Procedure{$PHI (N)$}{}
\State $\Phi[] \gets new Array[N]$
\For {$(p = 1 \text{; } p \leq N \text{; } p++)$}
\State $\Phi[p] \gets p$
\EndFor

\For {$(p = 2\text{; } p \leq N\text{; } p++)$}

\If {$\Phi[p] \neq p$} \Comment{$\Phi[p] \neq p \Leftrightarrow p\text{ não é primo}$}
\State \textbf{continue}
\EndIf

\For {$(n = p\text{; } n \leq N\text{; } n = n+p)$}
\State $\Phi[n] \gets \Phi[n] (\frac{p-1}{p})$
\EndFor

\EndFor

\State \Return {$\Phi[]$}
\EndProcedure
\end{algorithmic}
\end{algorithm}


\begin{corollary}\label{phi_potencia_nk}
$\Phi(n^k) = n^{k-1}\Phi(n)$, para inteiros positivos $n$ e $k$. 
\end{corollary}
\textbf{Demonstração:}
TODO


\subsection{Teorema de Euler}

\begin{theorem}[Teorema de Euler]\label{teorema_de_euler}
Dados números inteiros $a$ e $n$ primos entre si, temos que:
$a^{\Phi(n)} \equiv 1 (\bmod$ $n)$
\end{theorem}
\textbf{Demonstração:}
TODO usa residuos completo mod m



%----------------------------------------------------------------------------------------
%	SECTION
%----------------------------------------------------------------------------------------

\section{Sequência de Fibonacci}

\begin{definition}
A sequência de Fibonacci $Fib_n$ é uma sequência de números inteiros positivos em que cada termo subsequente corresponde a some dos dois termos anteriores.

\[
 Fib_n :=
  \begin{cases}
   0 & \text{se } n = 0 \\
   1 & \text{se } n = 1 \\
   Fib_{n-1} + Fib_{n-2} & \text{se } n \geq 2
  \end{cases}
\]
\end{definition}


\begin{corollary}\label{gcd_consecutivo_fib}
$MDC(Fib_n, Fib_{n-1}) = 1$, para $n \geq 2$
\end{corollary}
\textbf{Demonstração:}
Tome os primeiros termos da sequência de fibonacci: $1, 1, 2, 3, 5, 8,...$.
Claramente a expressão acima funciona para os primeiros termos.
Assuma que a expressão funciona para um inteiro qualquer $(k-1) > 2$ ($MDC(Fib_{k-1}, Fib_{k-2}) = 1$).

Provaremos por indução que a expressão sempre funciona.

$MDC(Fib_{k}, Fib_{k-1}) = MDC(Fib_{k-1} + Fib_{k-2}, Fib_{k-1})$

$MDC(Fib_{k-1} + Fib_{k-2}, Fib_{k-1}) = MDC(Fib_{k-2}, Fib_{k-1})$ ($\triangleright$ Pelo \textbf{Corolário} \autoref{corolario_gcd_soma})

Logo, temos que:

$MDC(Fib_{k}, Fib_{k-1}) = MDC(Fib_{k-2}, Fib_{k-1}) = 1$ $\square$



\begin{corollary}\label{gcd_combinacao_fib}
$Fib_{m+n} = Fib_mFib_{n+1} + Fib_{m-1}Fib_n$
\end{corollary}
\textbf{Demonstração:} Provaremos esse corolário por indução no índice $n$.

A base da indução será, $n=2$:

$Fib_{m+2} = Fib_m + Fim_{m+1} = Fib_m + Fib_m + Fib_{m-1}$

$Fib_{m+2} = 2Fib_m + 1Fib_{m-1} = Fib_mFib_{3} + Fib_{m-1}Fib_2$

Assumindo que a expressão funciona para todos os valores menores que $n$, temos:

$Fib_{m+n} = Fib_{m+n-2} + Fib_{m+n-1}$

$Fib_{m+n} = (Fib_{m}Fib_{n-1} + Fib_{m-1}Fib_{n-2}) + (Fib_{m}Fib_{n} + Fib_{m-1}Fib_{n-1})$

$Fib_{m+n} = Fib_m(Fib_{n-1} + Fib_{n}) + Fib_{m-1}(Fib_{n-2} + Fib_{n-1})$

$Fib_{m+n} = Fib_mFib_{n+1} + Fib_{m-1}Fib_n$ $\square$ 


\begin{theorem}\label{fibonacci_mdc}
$MDC(Fib_m, Fib_n) = Fib_{MDC(m, n)}, \forall m, n \in \mathbb{Z}$
\end{theorem}
\textbf{Demonstração:}

$MDC(Fib_m, Fib_n) = MDC(Fib_m, Fib_{qm + r})$ ($\triangleright$ \autoref{algoritmo_divisao}, $n = qm + r, 0 \leq r < n$)

$MDC(Fib_m, Fib_n) = MDC(Fib_m, Fib_{qm}Fib_{r+1} + Fib_{qm-1}Fib_{r})$ ($\triangleright$ \textbf{Corolário} \autoref{gcd_combinacao_fib}).

$MDC(Fib_m, Fib_n) = MDC(Fib_m, Fib_{qm-1}Fib_{r})$

Pelo \textbf{Corolário} \autoref{corolario_gcd_produto} e sabendo que $MDC(Fib_m,Fib_{qm-1})=1$, temos:

$MDC(Fib_m, Fib_n) = MDC(Fib_m, Fib_{r})$

$MDC(Fib_m, Fib_n) = MDC(Fib_m, Fib_{n \bmod m})$

Se tirarmos o símbolo funcional $Fib$, a última equação forma um passo do \textbf{Algoritmo de Euclides} ($MDC(m,n) = MDC(m, n \bmod m)$).

Podemos continuar esse processo até que o resto $r$ se torne $0$. O último resto não-nulo será
exatamente o Máximo Divisor Comum do dois números originais.

Desse modo, se aplicar-mos o \textbf{Algoritmo de Euclides} em $Fib_m$ e $Fib_n$ funciona da mesma maneira que se aplicar-mos aos índice $m$ e $n$.
E assim, ao chegarmos na base da recursão, $MDC(m,n) = MDC(s,0) = s$, teremos também: $MDC(Fib_m,Fib_n) = MDC(Fib_s,0) = Fib_s = Fib_{MDC(m,n)}$ $\square$.


\begin{theorem}
$Fib_n = \frac{\sqrt{5}}{5}((\frac{1+\sqrt{5}}{2})^n - (\frac{1-\sqrt{5}}{2})^n)$
\end{theorem}
\textbf{Demonstração:}

A demonstração asseguir foi baseada no livro: \textit{OLIVEIRA SANTOS, José Plínio de. Introdução à Teoria dos Números. IMPA, 1998. 85 p.}

$Fib_{n+1} = Fib_n + Fib_{n-1}$

$Fib_{n+1} - kFib_n = Fib_n + Fib_{n-1} - kFib_n$

$Fib_{n+1} - kFib_n = Fib_n + Fib_{n-1} - kFib_n + (kFib_{n-1}-kFib_{n-1}) + (k^2Fib_{n-1}-k^2Fib_{n-1})$

$Fib_{n+1} - kFib_n = (1 - k)(Fib_n - kFib_{n-1}) + (1 + k - k^2)Fib_{n-1}$

Se denotarmos as raízes de $k^2-k-1=0$ por $k_1$ e $k_2$, teremos que $k_1=\frac{1-\sqrt{5}}{2}$ e $k_2=\frac{1+\sqrt{5}}{2}$. 

$Fib_{n+1} - k_1Fib_b = k_2(Fib_n - k_1Fib_{n-1})$

$Fib_{n+1} - k_2Fib_b = k_1(Fib_n - k_2Fib_{n-1})$

Por iterações sucessivas dessas duas equações teremos que:

$Fib_{n+1} - k_1Fib_b = k_2^n(Fib_1 - k_1Fib_0) = k_2^n$

$Fib_{n+1} - k_2Fib_b = k_1^n(Fib_1 - k_2Fib_0) = k_1^n$

Subtraindo membro à membro nos dá:

$Fib_n(k_2 - k_1) = k_2^n - k_1^n$

$Fib_n = \frac{k_2^n - k_1^n}{k_2 - k_1}$

$Fib_n = \frac{(\frac{1+\sqrt{5}}{2})^n - (\frac{1-\sqrt{5}}{2})^n}{(\frac{1+\sqrt{5}}{2}) - (\frac{1-\sqrt{5}}{2})}$

$Fib_n = \frac{\sqrt{5}}{5}((\frac{1+\sqrt{5}}{2})^n - (\frac{1-\sqrt{5}}{2})^n)$ $\square$




%----------------------------------------------------------------------------------------
%	SECTION
%----------------------------------------------------------------------------------------

\section{Problemas Propostos}


%----------------------------------------------------------------------------------------
\subsection{UVA-11424}
\href{https://uva.onlinejudge.org/index.php?option=onlinejudge&page=show_problem&problem=2419}{11424 - GCD - Extreme (I)} \\

\textbf{Resumo:}
É dado um inteiro positivo $N$ ($1 < N < 200001$). O problema consiste em calcular o mais rápido possível a expressão:

$G(N) = \sum_{i=1}^{N-1}\sum_{j=i+1}^{N}MDC(i,j)$.
\\

\textbf{Solução:}
Trivialmente a expressão acima pode ser calculada em tempo proporcional à $O(n^2log(N))$, porém essa solução consome muito tempo e não será aceita no Judge Online. Vamos então mostrar uma solução mais eficiente.
\\

Primeiramente reescrevemos a expressão acima da seguinte maneira:

$G(N) = \sum_{j=2}^N\sum_{i=1}^{j-1}MDC(i,j)$ ( $\rhd$ Observe que as expressão são equivalentes).

Tome agora a função $F(M) = \sum_{i=1}^{M-1}MDC(i, M)$ $\Rightarrow$ $G(N) = \sum_{j=2}^NF(j)$.

Sabemos que todos os valores resultantes do método $MDC(i,M)$ calculados em $F(M)$ são divisores de $M$. Desse modo, podemos reescrever $F(M)$ da seguinte maneira:

$F(M) = \sum_{i=1}^{M-1}MDC(i, M) = \sum_{l=1}^{n}\lambda_l d_l$, em que, $d_1, d_2,..., d_n$ são os divisores de $M$, $\lambda_l$ é o número de vezes que o divisor $d_l$ aparece na somatória $\sum_{i=1}^{M-1}MDC(i,N)$, e $n$ é o número de divisores de $M$.
\\

Pelo Corolario \autoref{divisibilidade_mdc} temos que: $MDC(i,M) = d_l \Rightarrow MDC(i/d_l,M/d_l) = 1$. Logo o número de vezes que o divisor $d_l$ aparece na somatória, será igual ao número de primos entre si com $(M/d_l)$, ie, $\lambda_l = \Phi(M/d_l)$.

Reescrevendo novamente $F(M)$, temos:

$F(M) = \sum_{i=1}^{M-1}MDC(i, M) = \sum_{l=1}^n \lambda_l d_l = \sum_{l=1}^n \Phi(M/d_l) d_l$.

$G(N) = \sum_{j=2}^N \sum_{l=1}^n \Phi(j/d_l)d_l$ $\square$.
\\

\textbf{Pseudocódigo:}
\begin{algorithm}
\caption{GCD - Etreme(I)}\label{gcd_extreme}
\begin{algorithmic}[1]
\Procedure{G (N)}{}
\State $\Phi[] \gets PHI(N)$
\State $solution \gets 0$
\For {$j$ := $2$ to $N$}
\For {\textbf{each} divisor $d$ de $j$}
\State $solution \gets solution + \Phi[j/d] d$
\EndFor
\EndFor
\State \Return{$solution$}
\EndProcedure
\end{algorithmic}
\end{algorithm}


\textbf{Análise:}
O método $PHI(N)$ na linha 2 consome tempo proporcional à $O(N\sqrt{N})$.

O número de divisores de $j$ é proporcional à $O(\sqrt{N})$, já que $j \leq N$.

Assim a complexidade das linhas 4, 5, 6 do algoritmo é $O(N\sqrt{N})$.

Complexidade final do algoritmo: $O(N\sqrt{N})$.

\textbf{OBS.:} Para resolver o problema no Judge Online será preciso armazenar as soluções usando \href{https://linux.ime.usp.br/~stefanot/mac499/template.pdf}{Programação Dinâmica}.






%----------------------------------------------------------------------------------------
\subsection{TJU-3506}
\href{http://acm.tju.edu.cn/toj/showp3506.html}{3506 - Euler Function} \\

\textbf{Resumo:}
São dados três números positivos $n$, $m$ ($1 < n < 10^7$, $1 < m < 10^9$) e $d = 201004$.
O problema consiste em calcular a expressão: $\Phi(n^m) \bmod d$.
\\

\textbf{Solução:}
Pelo \textbf{Corolário} \autoref{phi_potencia_nk}, temos: 

$\Phi(n^m) \bmod d = (n^{m-1}\Phi(n)) \bmod d$

$\Phi(n^m) \bmod d = ((n^{m-1} \bmod d)(\Phi(n)) \bmod d) \bmod d)$

Desse modo, podemos calcular a primeiro fator do produto ($n^{m-1} \bmod d$) usando $EXPMOD()$ e a segundo fator com o método $PHI()$.
\\

\textbf{Pseudocódigo:}
\begin{algorithm}
\caption{Euler Functions}
\begin{algorithmic}[1]
\Procedure{$PhiEulerPotential (n, m, d)$}{}
\State $\Phi[] \gets PHI(n)$
\State $exp \gets EXPMOD(n, m-1, d)$
\State $solution \gets (exp$ $\Phi[n]) \bmod d$
\State \Return{$solution$}
\EndProcedure
\end{algorithmic}
\end{algorithm}

\textbf{Análise:}
As linhas $3$ e $4$ do algoritmo consomem tempo proporcional à $O(\log m)$ e $O(1)$ respectivamente.
Se precalcular-mos o vetor $\Phi[]$, temos que a complexidade total para calcular cada instância do problema será: $O(\log m)$ 



%----------------------------------------------------------------------------------------
\subsection{CodeChef-IITK2P05}
\href{https://www.codechef.com/problems/IITK2P05}{IITK2P05 - Factorization}\\

\textbf{Resumo:}
É dado um inteiro $N$ ($2 \leq N \leq 10^{18}$) e o valora de $\Phi(N)$.

O problema consiste em fatorizar $N$. 
\\

\textbf{Solução:}
A solução trivial para fatorar $N$ consome tempo proporcional à $O(\sqrt{N})$, porém para uma entrada na ordem de $10^{18}$ precisames de um algoritmo mais eficiente.

Se $N$ for primo, ie, $\Phi(N) = N-1$, já temos a solução.

Assuma então que $N$ é um número composto. Primeiro vamos iterar nos primeiros $\sqrt[3]{N}$ inteiros e remover todos os fatores primos de $N$ nesse intervalo. Tome $M$ como sendo o valor resultante.

Imagine que $M$ tenha três ou mais fatores primos. Sabemos que $M$ não tem nenhum fator primo menor que $\sqrt[3]{N}$, temos que: $M > (\sqrt[3]{N})^3 \Rightarrow M > N$ (contradição).
Desse modo, temos que $M$ tem no máximo dois fatores primos, e esses valores são maiores que $\sqrt[3]{N}$.

\begin{itemize}
  \item \textbf{Caso 1:} $M$ tem só um fator primo, ie, $M$ é primo:
		Basta checar se $\Phi(M) = M-1$

  \item \textbf{Caso 2:} $M$ tem dois fatores primos iguais, $M = p ^2$:
		Basta verificar se $M$ é um quadrado perfeito. Pode ser feito facilmente com busca binária.

  \item \textbf{Caso 3:} $M$ tem dois fatores primos distintos, $M = pq$:
		Se $M=pq$ então $\Phi(M) = (p-1)(q-1)$. Temos então um sistema com duas equações e duas incógnitas. Se resolvermos o sistema encontraremos a fatoração de $M$ e assim a fatoração de $N$.
\end{itemize}
O único problema agora é calcular $\Phi(M)$ a partir de $\Phi(N)$. Assuma que $N = p_1^{a_1}p_2^{a_2}...p_k^{a_k}M$, com $k \geq 0$ e 
 $p_i$ os fatores primos distintos de $N$ removidos na primeira etapa do algoritmo. Temos então:

$N = p_1^{a_1}p_2^{a_2}...p_k^{a_k}M \Rightarrow \Phi(N) = \Phi(p_1^{a_1}p_2^{a_2}...p_k^{a_k}M)$

$N = p_1^{a_1}p_2^{a_2}...p_k^{a_k}M \Rightarrow \Phi(N) = \Phi(p_1^{a_1})\Phi(p_2^{a_2})...\Phi(p_k^{a_k})\Phi(M)$ ($\triangleright$ \autoref{phi_multiplicativa})

$\Rightarrow \Phi(N) = (p_1^{a_1}-p_1^{a_1-1})(p_2^{a_2} - p_2^{a_2-1})...(p_k^{a_k}-p_k^{a_k-1})\Phi(M)$ ($\triangleright$ \autoref{phi_potencia})

$\Rightarrow \Phi(M) = \frac{\Phi(N)}{(p_1^{a_1}-p_1^{a_1-1})(p_2^{a_2} - p_2^{a_2-1})...(p_k^{a_k}-p_k^{a_k-1})}$
\\

\textbf{Pseudocódigo:}
\begin{algorithm}
\caption{Fatoração de $N$}
\begin{algorithmic}[1]
\Procedure{$Factorization (N, \Phi_N)$}{}
\State $S \gets \emptyset$ \Comment{$S$ contém os fatores primos de $N$}
\State $M \gets N$
\State $\Phi_M \gets \Phi_N$
\\
\If {$\Phi_N = N-1$} \Comment{Se $N$ for primo}
\State $S \gets S \cup \{N\}$
\State \Return {$S$}
\EndIf
\\
\For {\textbf{each} $p \text{ primo menor igual à } \sqrt[3]{N}$}
\While {$M \equiv 0 (mod $ $p)$}
\State $M \gets \frac{M}{p}$
\State $S \gets S \cup \{p\}$
\EndWhile
\EndFor
\\
\For {\textbf{each} $p \in S$}
\State $\Phi_M \gets \frac{\Phi_m}{p^a - p^{a-1}}$ \Comment{$a :=$ número de vezes que o primo $p$ é inserido em $S$}
\EndFor
\\
\If {$\Phi_M = M-1$} \Comment{Se $M$ for primo}
\State $S \gets S \cup \{M\}$
\State \Return {$S$}
\EndIf
\\
\State $(p, q) \gets System(M, \Phi_M)$ \Comment{Resolve o sistema de 2 equações e 2 incógnitas}
\State $S \gets S \cup \{p, q\}$
\State \Return {$S$}

\EndProcedure
\end{algorithmic}
\end{algorithm}


\textbf{Análise:}
O laço da linha $10$ consome tempo proporcional à $O(\sqrt[3]N)$. Já o laço da linha $11$ consome tempo proporcional à $O(\log_pN)$, pois tem no máximo $a$ iterações ($a$ é o número de vezes que o fator primo $p$ aparece em $N$) e assim, $p^a < N \Rightarrow a < \log_p{N}$. 

As linhas 15-16 rodam em $O(log_pN)$, já que o número máximo de elementos distintos em $S$ é $log_pN$ ($p$ é o menor primo que divide $N$).
E as linhas 18-23 rodam em $O(1)$.

Assim, o algoritmo total consome tempo proporcional à $O(\sqrt[3]N \log N)$. 
Obeserve que esse algoritmo é bem mais eficiente que o algoritmo trivial para fatoração $O(\sqrt N)$.


%----------------------------------------------------------------------------------------
\subsection{CodeChef-PUPPYGCD}
\href{https://www.codechef.com/problems/PUPPYGCD}{PUPPYGCD - Puppy and GCD}\\


\textbf{Resumo:}
São dados inteiros positivos $N$ e $D$. O problema consite em calcular o número de pares não-ordenados $\{A,B\}$, tal que $1\leq A,B \leq N$ e $MDC(A,B)=D$.
\\

\textbf{Solução:}
Claramente se $D$ for maior que $N$, não há nenhum par que satisfaz as condições. Logo, assumiremos que $D \leq N$.

Pelo \textbf{Corolário} \autoref{divisibilidade_mdc} sabemos que se $MDC(A,B)=D$ então $MDC(\frac{A}{D}, \frac{B}{D})=1$. Assim, podemos reduzir o problema
em calcular o número de pares não-ordenados $\{A,B\}$, tal que $1\leq A,B \leq \frac{N}{D}$ e $MDC(A,B)=1$.
Logo o número de pares será igual a somatória de $\Phi(r)$, com $1 \leq r \leq \frac{N}{D}$, já que $\Phi(r)$ nos dá a qunatidade de números menores ou iguais a $r$ e primo com entre si com $r$. Observe que essa somatória nos dá somente a metade do número de pares, já que o problema consiste em calcular pares não-ordenados.
\\

\textbf{Pseudocódigo:}
\begin{algorithm}
\caption{Puppy and GCD}
\begin{algorithmic}[1]
\Procedure{$CalculatePairs (N, P)$}{}
\If {$D > N$}
\State \Return $0$
\EndIf
\\
\State $\Phi[] \gets PHI(\frac{N}{D})$
\State $count \gets 0$
\\
\For {($A = 1 \text{; } A \leq \frac{N}{D} \text{; } A++)$}
\State $count \gets count + \Phi[A]$
\EndFor
\\
\State $count \gets 2$ $count - 1$
\State \Return $count$
\EndProcedure
\end{algorithmic}
\end{algorithm}

\textbf{Análise:}
Se precalcular $\Phi[]$ teremos que o fator limitando do algoritmo será o laço da linha $8$, e a complexidade do algoritmo será $O(\frac{N}{D})$.

Obs.: Quando dobramos o valor de $count$ na linha $11$, tem um único $\{1,1\}$ que é contado duas vezes, e por isso precisamos subtrair $1$ do valor final. 
Esse par corresponde ao par $\{D,D\}$ do problema original.




%----------------------------------------------------------------------------------------
\subsection{CodeChef-MODEFB}
\href{https://www.codechef.com/problems/MOREFB}{71544 - Another Fibonacci}\\

\textbf{Resumo:}
São dados dois números inteiros $N$, $K$ ($1 \leq N \leq 50000$, $1 \leq K \leq N$) e um conjunto $S \subset \mathbb{N}$ com $N$ elementos, tal que, $\forall s \in S, 1 \leq s \leq 10^9$.

Tome a seguinte função:

$F(S) = \sum_{A \subset S \hspace{1mm} e\hspace{1mm} |A| = K}^{} Fib(sum(A))$, onde $sum(A) = \sum_{a \in A}a$. %$\Phi(n^m) \bmod 201004.$

O problema consiste em calcular a expressão:  
$F(S) \bmod 99991 $
\\

\textbf{Solução:}
\\

\textbf{Pseudocódigo:}
\begin{algorithm}
\caption{Another Fibonacci}
\begin{algorithmic}[1]
\Procedure{F (S)}{}

\EndProcedure
\end{algorithmic}
\end{algorithm}


\textbf{Análise:}







%----------------------------------------------------------------------------------------
\subsection{UVA-10311}
\href{https://uva.onlinejudge.org/index.php?option=onlinejudge&page=show_problem&problem=1252}{10311 - Goldbach and Euler}\\

\textbf{Resumo:}
É dado um número inteiro $n$ ($0 < n \leq 10^8$). O problema consite em verificar se $n$ pode, ou não pode, ser escrito como a soma de dois números primos. 
E em caso afirmativo encontrar o valor desses dois primos.
\\

\textbf{Solução:}
\\

\textbf{Pseudocódigo:}
\begin{algorithm}
\caption{Goldbach and Euler}
\begin{algorithmic}[1]
\Procedure{FindTwoPrimesSum (n)}{}

\EndProcedure
\end{algorithmic}
\end{algorithm}


\textbf{Análise:}



%----------------------------------------------------------------------------------------
\subsection{Codeforces-227E}
\href{http://codeforces.com/contest/227/problem/E}{227E - Anniversary}\\

\textbf{Resumo:}
\\

\textbf{Solução:}
\\

\textbf{Pseudocódigo:}
\begin{algorithm}
\caption{Anniversary}
\begin{algorithmic}[1]
\Procedure{FindTwoPrimesSum (n)}{}

\EndProcedure
\end{algorithmic}
\end{algorithm}


\textbf{Análise:}


