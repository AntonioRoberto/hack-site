% Chapter Template

\chapter{Conclusão} % Main chapter title

\label{Chapter5} % Change X to a consecutive number; for referencing this chapter elsewhere, use \ref{ChapterX}

%----------------------------------------------------------------------------------------
%	SECTION 1
%----------------------------------------------------------------------------------------

Neste trabalho mostramos como resultados interessantes de Teoria dos
Números, como, por exemplo, o Teorema de Bézout,  Teorema de Fermat,
Teorema de Wilson ou função totiente de Euler podem surgir em
aplicações reais. Nossa abordagem foi explorar problemas de diversas
competições de programação, como o ACM ICPC e o Codeforces, em que o
conhecimento dos resultados de Teoria dos Números faz a diferença
entre resolver o problema ou tentar uma abordagem incorreta, como a
força bruta.

Buscamos explorar os conceitos de busca modular, dividindo os
resultados e respectivos problemas, em 3 grupos principais:
Divisibilidade, Aritmética Modular e Funções Aritméticas. O leitor
interessado é convidado a explorar novos conceitos e incrementar o
conteúdo deste texto com novas aplicações de resultados interessantes
de Teoria dos Números.

Este texto é um bom exemplo em que se mostra de forma clara e evidente
aplicações importantes em Computação de áreas que, a princípio, podem
parecer distantes para os alunos. Outros exemplos deste tipo poderiam
servir de motivação para estudantes de graduação para cursar
disciplinas como Álgebra, Álgebra Linear ou Física.


%Esse trabalho vem preencher um pouco da falta de um bom material didático sobre Teoria dos Números aplicada em Competições de Programação.

%Como foi idealizado na concepção desse projeto, conseguimos criar um trabalho que mostra a aplicação direta dessa teoria na resolução de problemas complexos,
%além de ilustrar algumas técnicas de programação.

%Pelo fato de ser bem modular, é possível incrementar o conteúdo do mesmo com problemas e teorias de maneira coerente, sendo esse o próximo passo para esse trabalho.
