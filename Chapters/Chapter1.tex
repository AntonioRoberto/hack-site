% Chapter 1

\chapter{Divisibilidade} % Main chapter title

\label{Chapter1} % Change X to a consecutive number; for referencing this chapter elsewhere, use \ref{ChapterX}

%----------------------------------------------------------------------------------------
%	SECTION 1
%----------------------------------------------------------------------------------------

\section{Introdução}

Nesse seção vamos descrever algumas definições e propriedades dos números inteiros que serão utilizados ao longo desse trabalho.

\begin{definition}
\end{definition}

\begin{corollary}\label{corolario_divisibilidade_1}
Dado um subconjunto dos inteiros $S = \{S_1, S_2, S_3, ..., S_n\}$ ordenado crescentemente, e um número inteiro $d$, tal que, $d|(S_i-S_{i-1})$, $2 \leq i \leq n$. 

Nessas Condições temos que: $d|(S_i-S_j)$, $\forall S_i, S_j \in S$.
\end{corollary}
\textbf{Demonstração:}
Tome $S_i,S_j \in S$ quaisquer, e sem perda de generalidade assuma que $S_i \geq S_j$ (ie, $i \geq j$, pois $S$ está ordenado crescentemente).

Como $i \geq j$, tome $r \in \mathbb{N}$ como sendo a diferença entre $i$ e $j$ : $i = j + r$.

Vamos agora provar por indução que $d|(S_{j+r}-S_j)$.

Para $r=0$ ou $r=1$ a demostração segue trivialmente.

Assuma que o corolário funciona para $(r-1)$, ie, $d|(S_{j+r-1}-S_j)$. 

Temos então que: 

$d|(S_{j+r}-S_{j+r-1}) \Rightarrow d|(S_{j+r}-S_{j+r-1})+(S_{j+r-1}-S_j) \Rightarrow d|(S_{j+r}-S_j)$ 


\begin{corollary}
O \textbf{Corolário} \autoref{corolario_divisibilidade_1} funciona mesmo se o conjunto $S$ não estiver ordenado.
\end{corollary}
\textbf{Demonstração:}
Deixaremos a demostração a cargo do leitor.


\begin{theorem}[Teorema da divisão]\label{algoritmo_divisao}
Para todo número inteiro $a$ e qualquer número inteiro positivo $n$, existe inteiros únicos $q$ e $r$, tal que:

$a = qn + r$, $0 \leq r < n$

O valor $q$ ($q = \lfloor  \frac{a}{n} \rfloor$) é chamado de \textbf{quociente} da divisão, e o valor $r$ ($r = a \bmod n$) é chamado de \textbf{resto}
(ou \textbf{resíduo}) da divisão.
\end{theorem}
\textbf{Demonstração:}
Suponha que $q$ e $r$ não sejam únicos, ie, que exista $q^*$ e $r^*$ tal que: $a = q^*n + r^*, 0 \leq r^* < n$.

$a = qn + r = q^*n + r^* \Rightarrow (r - r^*) = (q^* - q)n \Rightarrow (r - r^*) \equiv (q^* - q)n \equiv 0 (mod$ $n)$

Porém, como $r \neq r^*$, e tanto $r$ quanto $r^*$ são menores que $n$, temos que: 

$r \not\equiv r^* (mod$ $n) \Rightarrow (r - r^*) \not\equiv 0 (mod$ $n)$

Chegando numa contradição, e assim $q$ e $r$ são únicos $\square$ \\

%-----------------------------------
%	SECTION 2
%-----------------------------------
\section{Números Primos}

\begin{definition} 
Todo número inteiro n (n > 1) que têm apenas dois divisores distintos (1 e n) é chamado de número primo. Se n (n > 1) não for primo, dizemos que n é número composto.
\end{definition}

\begin{theorem}[Fatoração Única]\label{fatoracao_unica}
Um número natural qualquer $n$, pode ser escrito unicamente como um produto da forma: 
$n = p_1^{a_1}p_2^{a_2}...p_k^{a_k}$, onde os $p_i$ são números primos, $p_1 < p_2 < ... < p_k$, e os números $a_i$ são inteiros positivos.
\end{theorem}
\textbf{Demonstração:}
Deixaremos a demostração a cargo do leitor.
\textbf{Dica:} Use o fato de que o conjunto dos primos que divide um número inteiro é único, e fato de que se qualquer potência $a_i$ for alterado o valor de $n$ será alterado.


%-----------------------------------
%	SECTION 3
%-----------------------------------
\section{Máximo Divisor Comum}

\begin{definition}
O Máximo Divisor Comum de dois inteiros quaisquer $a$ e $b$ (com a ou b diferente de zero), denotado por $MDC(a,b)$, é o maior inteiro que divide ambos $a$ e $b$.
\end{definition}


\begin{corollary}\label{gcd_modular}
Para números inteiros quaisquer $a$ e $b$, $MDC(a,b) = MDC(b, a \bmod b)$
\end{corollary}
\textbf{Demonstração:}
TODO


\begin{corollary}\label{divisibilidade_mdc}
$MDC(a,b) = d \Rightarrow MDC(a/d, b/d) = 1$
\end{corollary}
\textbf{Demonstração:}
TODO


\subsection{Algoritmo de Euclides}
A ideia principal do \textbf{Algoritmo de Euclides} é calcular recursivamente o \textbf{Máximo Divisor Comum} de dois números baseando-se no 
\textbf{Corolário} \autoref{gcd_modular}.\\

\textbf{Pseudocódigo:}
\begin{algorithm}
\caption{Algoritmo de Euclides}\label{mdc}
\begin{algorithmic}[1]
\Procedure{$MDC (a, b)$}{}
\If {$b = 0$}
\State \Return $a$
\Else
\State \Return $MDC(b, a \bmod b)$
\EndIf
\EndProcedure
\end{algorithmic}
\end{algorithm}




\subsection{Teorema de Bézout}

\begin{theorem}[Teorema de Bézout]\label{teorema_bezout}
$\forall$ $a$, $b \in \mathbb{Z}$, $\exists$ $x, y \in \mathbb{Z} \mid ax + by = mdc(a, b).$
\end{theorem}
\textbf{Demonstração:}
De acordo com \autoref{teorema_bezout}




\begin{corollary}\label{corolario_gcd_soma}
Para números inteiros quaisquer $a$ e $b$, $MDC(a,b) = MDC(a,a \pm b)$
\end{corollary}
\textbf{Demonstração:}
A prova dessa expressão vem do fato de que qualder divisor de $a$ e $b$, é também divisor de $(a \pm b)$.



\begin{corollary}\label{corolario_gcd_produto}
Para números inteiros quaisquer $a$ e $b$, temos:

$MDC(a,b) = 1 \Rightarrow MDC(a,bk) = MDC(a,k)$, com $k \in \mathbb{Z}$
\end{corollary}
\textbf{Demonstração:}
A prova dessa expressão vem do fato de que qualder divisor $d$ de $a$ e $bk$, é também divisor de $k$, pois $d$ não divide $b$ ($MDC(a,b) = 1$).


%----------------------------------------------------------------------------------------
%	SECTION
%----------------------------------------------------------------------------------------

\section{Crivo de Erastóteles}





%----------------------------------------------------------------------------------------
%	SECTION
%----------------------------------------------------------------------------------------

\section{Problemas Propostos}


%----------------------------------------------------------------------------------------
\subsection{UVA-10407}
\href{https://uva.onlinejudge.org/index.php?option=onlinejudge&page=show_problem&problem=1348}{10407 - Simple Division} \\

\textbf{Resumo:} 
Tome $P(S) := \{ x \in \mathbb{Z} \mid  \forall a , b \in S , a \equiv b ( mod$ $x)\}$ em que $S \subset \mathbb{Z}$.

O problema consiste em encontrar o valor máximo de $P(S)$ dado um conjunto $S$.
\\

\textbf{Solução:} 
Seja $S = \{S_1, S_2, S_3, ..., S_n\}$, com $n = |S|$, o conjunto dado pelo problema (assumiremos que os valores de S estão ordenados crescentemente).

Tome um número qualquer $d \in P(S)$. Por definição temos que $\forall S_i, S_j \in S$, $S_i \equiv S_j ( mod$ $d) \Rightarrow $ 
$ (S_i-S_j) \equiv 0 ( mod$ $d) \Rightarrow d \mid (S_i-S_j)$ .

Pelo \textbf{Corolário} \autoref{corolario_divisibilidade_1} sabemos que:

$d | (S_i-S_{i-1})$, $ \forall i \in \mathbb{N}, 2 \leq i \leq n \Rightarrow d | (S_i-S_j)$, $ \forall S_i, S_j \in S \Rightarrow d \in P(S)$.

E desse modo, para calcular o valor máximo de $P(S)$ só precisamos calcular o Máximo Divisor Comum das diferenças $(S_i-S_{i-1})$ com $i$ variando de $2$ à $n$ $\square$.
\\

\textbf{Pseudocódigo:}
\begin{algorithm}
\caption{Simple Division}\label{euclid}
\begin{algorithmic}[1]
\Procedure{GetMaximumValue (S)}{}
\State $S \gets sort(S)$ \Comment{sort(X) retorna o conjunto X ordenado.} 
\State $maxValue \gets 0$
\For {i := 2 to |S|} 
\State $maxValue \gets MDC(maxValue, S_i - S_{i-1})$
\EndFor
\State \Return{$maxValue$}
\EndProcedure
\end{algorithmic}
\end{algorithm}


